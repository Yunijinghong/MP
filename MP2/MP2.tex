%!TEX program = xelatex
\documentclass[dvipsnames, svgnames,a4paper,11pt]{article}
\usepackage{tikz}
\usetikzlibrary{calc}
\usepackage{eso-pic}
\AddToShipoutPictureBG{%
\begin{tikzpicture}[overlay,remember picture]
\draw[line width=0.6pt] % 边框粗细
    ($ (current page.north west) + (0.6cm,-0.6cm) $)
    rectangle
    ($ (current page.south east) + (-0.6cm,0.6cm) $); % 边框位置
\end{tikzpicture}}

\usepackage{xcolor}
\definecolor{c1}{HTML}{086173} % 目录颜色
\definecolor{c2}{HTML}{E20129} % 引用颜色

\usepackage{ctex}
\usepackage[top=28mm,bottom=28mm,left=15mm,right=15mm]{geometry}
\usepackage{hyperref}
\hypersetup{
	colorlinks,
	linktoc = section,
	linkcolor = c1,
	citecolor = c1
}
\usepackage{amsmath,enumerate,multirow,float}
\usepackage{tabularx}
\usepackage{tabu}
\usepackage{subfig}
\usepackage{fancyhdr}
\usepackage{graphicx}
\usepackage{wrapfig}
\usepackage{physics}
\usepackage{appendix}
\usepackage{amsfonts}
\usepackage{makecell}
\usepackage{tcolorbox}
\tcbuselibrary{skins,breakable}
\newtcolorbox{tbox}[2][]{
    colframe=black!70!,
    breakable,
    enhanced,
	boxrule =0.5pt,
    title = {#2},
    fonttitle = \large\kaishu\bfseries,
	drop fuzzy shadow,
    #1
}
\newtcolorbox[auto counter,number within=section]{question}[1][]{
  top=2pt,bottom=2pt,arc=1mm,
  boxrule=0.5pt,
  breakable,
  enhanced,
  coltitle=c1!80!gray,
  colframe=c1,
  colback=c1!3!white,
  drop fuzzy shadow,
  title={思考题~\thetcbcounter:\quad},
  fonttitle=\bfseries,
  attach title to upper,
  #1
}

% ---------------------------------------------------------------------
\usepackage{cleveref}
\crefformat{figure}{#2{\textcolor{c2}{Figure #1}}#3}
\crefformat{equation}{#2{(\textcolor{c2}{#1})}#3}
\crefformat{table}{#2{\textcolor{c2}{Table #1}}#3}

% ---------------------------------------------------------------------
\fancypagestyle{plain}{\pagestyle{fancy}}
\pagestyle{fancy}
\lhead{\kaishu 中山大学物理与天文学院近代物理实验\uppercase\expandafter{\romannumeral1}}
\rhead{\kaishu 实验报告By黄罗琳}
\cfoot{\thepage}

% ---------------------------------------------------------------------
\renewcommand{\contentsname}{\centerline{\huge 目录}}
\usepackage{titlesec}
\usepackage{titletoc}
\titleformat{\section}{\centering\LARGE\songti}{}{1em}{}

% ---------------------------------------------------------------------
\usepackage{listings}
\lstloadlanguages{python}
\lstdefinestyle{pythonstyle}{
backgroundcolor=\color{gray!5},
language=python,
frameround=tftt,
frame=shadowbox, 
keepspaces=true,
breaklines,
columns=spaceflexible,                   
basicstyle=\ttfamily\small,
keywordstyle=[1]\color{c1}\bfseries, 
keywordstyle=[2]\color{Red!70!black},   
stringstyle=\color{Purple},       
showstringspaces=false,
commentstyle=\ttfamily\scriptsize\color{green!40!black},
tabsize=2,
morekeywords={as},
morekeywords=[2]{np, plt, sp},
numbers=left,
numberstyle=\it\tiny\color{gray},
stepnumber=1,
rulesepcolor=\color{gray!30!white}
}

% ---------------------------------------------------------------------
\def\degree{${}^{\circ}$} % 角度
\graphicspath{{./images/}} % 插入图片的相对路径
\allowdisplaybreaks[4]  % 允许公式跨页 
\usepackage{lipsum}
\usepackage{adjustbox}

\usepackage{multirow}
\usepackage{booktabs} % 如果需要更好看的表格
\usepackage{caption}  % 处理标题
\usepackage{array}    % 处理表格的列格式
\usepackage{graphicx} % 如果要插入图片
%\usepackage{mathrsfs} % 字体
%\captionsetup[figure]{name=Figure} % 图片形式
%\captionsetup[table]{name=Table} % 表格形式
\begin{document}
	
	% 实验报告封面	
	% 顶栏
	\begin{table}
		\renewcommand\arraystretch{1.7}
		\begin{tabularx}{\textwidth}{
				|X|X|X|X
				|X|X|X|X|}
			\hline
			\multicolumn{2}{|c|}{预习报告}&\multicolumn{2}{|c|}{实验记录}&\multicolumn{2}{|c|}{分析讨论}&\multicolumn{2}{|c|}{总成绩}\\
			\hline
			\LARGE25 & & \LARGE25 & & \LARGE30 & & \LARGE80 & \\
			\hline
		\end{tabularx}
	\end{table}
	% ---
	
	% 信息栏
	\begin{table}
		\renewcommand\arraystretch{1.7}
		\begin{tabularx}{\textwidth}{|X|X|X|X|}
			\hline
			年级、专业: & 2022级 物理学 &组号: & \\
			\hline
			姓名: & 黄罗琳、王显   & 学号: &  22344001、22344002 \\
			\hline
			实验时间: & 2024/9/20 & 教师签名: & \\
			\hline
		\end{tabularx}
	\end{table}
	% ---
	
	% 大标题
	\begin{center}
		\LARGE D1 \quad 锁相放大器与弱信号测量
	\end{center}
	% ---
	
	% 注意事项
	
	% 基本
	\textbf{【实验报告注意事项】}
	
		\begin{enumerate}
			\item 预习报告:课前认真研读实验讲义,弄清实验原理;实验所需的仪器设备、用具及其使用、完成课前预习思考题;了解实验需要测量的物理量,并根据要求提前准备实验记录表格。
			\item 实验记录:认真、客观记录实验条件、实验过程中的现象以及数据。实验记录请用珠笔或者钢笔书写并签名(\textcolor{red}{\textbf{用铅笔记录的被认为无效}})。\textcolor{red}{\textbf{保持原始记录,包括写错删除部分,如因误记需要修改记录,必须按规范修改。}}(不得输入电脑打印,但可扫描手记后打印扫描件);离开前请实验教师检查记录并签名。
			\item 数据处理及分析讨论:处理实验原始数据(学习仪器使用类型的实验除外),对数据的可靠性和合理性进行分析;按规范呈现数据和结果(图、表),包括数据、图表按顺序编号及其引用;分析物理现象(含回答实验思考题,写出问题思考过程,必要时按规范引用数据);最后得出结论。
		\end{enumerate}
		
		
	
		
	

	
	% 安全
		
	 \textbf{本实验报告阅读说明}:
		\begin{enumerate}
			\item 本实验报告基于基本实验原理,尽可能符合逻辑,并且保证实验\textbf{所有要求的内容均有结果或解释},如果出现超出能力范围,会进行理论建模进行相关说明。
			\item 本实验报告由组内人员共同完成,所有分工合作均保证工作量尽可能平均,如有特殊情况会在结语处进行说明。
			\item 本实验报告会于实验课程结束后上传至GitHub开源,相关仓库包括基础物理实验、电子技术实验,可供查阅。
		\end{enumerate}
	% 目录
	\clearpage
	\tableofcontents
	\clearpage
	% ---
	
	
	
	% 预习报告	
	
	% 小标题
	\setcounter{section}{0}
	\section{D2 材料真空兼容性测试和等离子特性研究 \quad\heiti 原理背景}
	% ---
	
	% 实验目的
	\subsection{实验目的}
	
\begin{enumerate}
	\item 学习基本的真空知识和技术, 掌握真空的获得和测量方法。
	\item 通过真空气体放电实验, 验证帕邢定律, 了解气体放电基本物理过程。
	\item 利用光纤光谱仪研究真空气体放电等离子体光谱特性, 获得等离子体基本参数, 了
	解等离子体物理的基本知识。
	\item 了解四极质谱仪工作原理, 使用四极质谱仪进行真空系统检漏和气体成分分析。
\end{enumerate}

\subsection{实验内容}

\begin{itemize}
    \item \textbf{高真空的获得}
    \begin{itemize}
        \item 使用机械泵和分子泵获得高真空。
        
        \item \textbf{1.1 启动机械泵观察记录}
        \begin{itemize}
            \item 启动机械泵并观察记录真空度(真空计压强)随时间(5分钟)的变化。
            \item 机械泵先抽真空压强低于 10 Pa 后,启动分子泵观察记录真空度随时间的变化,待分子泵达到额定转速(约需 8分钟)后再观察记录(5分钟)。
        \end{itemize}
        
        \item \textbf{1.2 停止分子泵观察记录}
        \begin{itemize}
            \item 停止分子泵观察记录真空度(真空计压强)随时间的变化(分子泵转速降为零约需 8分钟)。
            \item 待分子泵完全停止后,关闭机械泵,记录真空度(真空计压强)随时间的变化(5分钟)。
        \end{itemize}
        
        \item \textbf{思考:} 真空度(真空计压强)随时间变化反映了真空系统的什么特性?\\
        \textit{答:} 真空度随时间的变化反映了真空系统的泄漏特性和泵的性能,能够指示系统的密封性以及气体分子的动态行为。
    \end{itemize}

    \item \textbf{气体放电实验}
    \begin{itemize}
        \item 通过真空气体放电实验,测量击穿电压与电极间隙和气压之间的关系,验证帕邢定律,了解气体放电基本物理过程。
    \end{itemize}

    \item \textbf{气体放电现象观察}
    \begin{itemize}
        \item 观察气体放电(发光)现象,利用光纤光谱仪研究气体放电产生的等离子体光谱特性,获得等离子体基本参数,了解等离子体物理的基本知识。
        
        \item \textbf{思考:} 等离子体光谱反映了等离子体的什么特性?能得到等离子体的什么参数?\\
        \textit{答:} 等离子体光谱反映了等离子体的电子能级跃迁和离子化状态,能够得到等离子体的温度、密度以及成分信息。
    \end{itemize}

    \item \textbf{四极质谱仪}
    \begin{itemize}
        \item 了解四极质谱仪工作原理,使用四极质谱仪进行真空系统检漏和气体成分分析。
        
        \item \textbf{探讨:} 研究四极电场特性及其中离子运动方程,深入探讨四极质谱仪工作原理。\\
        \textit{答:} 四极质谱仪通过电场控制离子的稳定性,只有特定质量的离子能在电场中稳定运动,研究其运动方程可帮助理解其工作原理和性能。
    \end{itemize}
\end{itemize}


\subsection{仪器用具}
上海宜准公司 VQP01 真空平台。 (针对帕邢实验、 四极质谱实验等实验项目而设计的
一台综合实验装置, 该装置由真空放电腔体、 机械泵、 分子泵、 高压电源、 四极质谱仪、 真
空计以及击穿电压测量系统等装置构成。 )

\subsection{ 实验安全注意事项}

\begin{itemize}
    \item 操作前请检查真空腔体是否密封,检查高压电源开关、分子泵电源开关是否断开,以及应急按钮是否断开。
    
    \item 注意高电压电源使用安全。 
    \begin{itemize}
        \item 高压电源受真空计控制,实验前请确认真空计是否通电;
        \item 通电情况下请勿插拔高压电源后面板高压输出接口,切勿接触后侧电力控制部分;
        \item 实验前请检查高压电源调节旋钮,务必置零;
        \item 实验过程中请勿接触高压电源后面板以及高压电源内侧结构。
    \end{itemize}
    
    \item 若实验中用到分子泵,需机械泵先抽真空压强低于 10 Pa 以下才能开启分子泵电源。
    
    \item 若实验中用到四极质谱仪,开启四极质谱仪时保证真空压强低于 5.0E-2 Pa。
\end{itemize}
\subsection{原理概述}
\subsubsection{真空的获得与测量}
在给定空间内,气体压强低于一个大气压的气体状态,称之为真空。真空的获得就是人们常说的“抽真
空”,即利用各种真空泵将被抽容器中的气体抽出,使该空间的压强低于一个大气压。真空测量是指用特定
的仪器和装置,对某一特定空间内真空高低的测定,这种仪器或装置称为真空计(仪器、规管)。
\subsubsection{固体对气体的吸附及气体的脱附}
气体吸附就是固体表面捕获气体分子的现象,吸附分为物理吸附和化学吸附。其中物理吸附没有选择性,
任何气体在固体表面均可发生,主要靠分子间的相互吸引力引起的。物理吸附的气体容易发生脱附,而且这
种吸附只在低温下有效;化学吸附则发生在较高的温度下,与化学反应相似,气体不易脱附,但只有当气体
和固体表面原子接触生成化合物时才能产生吸附作用。气体的脱附是气体吸附的逆过程。通常把吸附在固体
表面的气体分子从固体表面被释放出来的过程叫做气体的脱附。
\subsubsection{气体放电、等离子体和帕邢定律}
气体放电的基本过程是利用外(电)场加速电子使之碰撞中性原子(分子)来电离气体。等离子体由离
子、电子以及未电离的中性原子(分子)的集合组成,整体呈中性的物质状态。气体放电是产生等离子体的
一种常见形式。帕邢定律是表征均匀电场气体间隙击穿电压、间隙距离和气压间关系的定律。
\subsubsection{四极质谱仪}
四极杆上加有直流和射频交流分量电压(势),使得一定质量电荷比的离子可稳定的通过四极杆质量过
滤器(离子能够稳定地通过四极电场),而不会撞上或逸出四极杆,可将离子根据质量电荷比进行过滤分类,
归纳成质谱。
\begin{question}
	真空物理学的研究内容及方法?
\end{question}
空物理学是真空科学与技术的理论基础,是研究稀薄气体物理运动规律的理论,主要采用统计物理
和热力学方法研究稀薄气体分子运动以及气体分子之间、分子与器壁之间的相互作用。
\begin{question}
	真空的定义? 理想气体压强公式? 气体分子的平均自由程?
\end{question}
真空的定义:在给定空间内,气体压强低于一个标准大气压的气体状态(气体分子的密度低于标准大
气压的气体分子密度的状态),称之为真空。

理想气体压强公式:$ p = nkT$。 n 为气体分子数密度, k 为玻尔兹曼常数, T 为热力学温度。混合气体
的压强等于各组分的分压强之和(道尔顿分压定律)。
\begin{question}
	真空气体放电的基本过程?帕邢定律?
\end{question}
真空气体放电的基本物理过程是利用外电场加速电子使之碰撞中性原子(分子)来电离气体,主要包
括粒子的激发、电离、复合、漂移、扩散等基本过程。

在(平面电极)均匀电场中,气体击穿电压是气体压力 p 与电极距离 d 乘积的函数,通称为帕邢定律,
其特点为:在一定 pd 数值,击穿电压有极小值。
\begin{question}
	辉光放电的特点?
	\end{question}
	辉光放电是指低压气体中显示辉光的气体放电现象,即是稀薄气体中的自持放电现象。两个平行电极板间的电压(电场)加速电子将中性原子或分子激发,而被激发的粒子由激发态降回基态时会以光的形式释放出能量。放电的整个通道由不同亮度的区间组成,即由阴极表面开始,依次为:阿斯通暗区;阴极光层;阴极暗区(克鲁克斯暗区);负辉光区;法拉第暗区;正柱区;阳极暗区;阳极光层。这些光区是空间电离过程及电荷分布所造成的结果,与气体类别、气体压力、电极材料等因素有关。
	
	\begin{question}
	等离子体是什么?等离子体的基本参数?
	\end{question}
	等离子体(Plasma)由离子、电子以及中性原子(分子)的集合组成,整体宏观呈中性的物质状态。等离子体的基本参数如下:
	\begin{itemize}
		\item 电离度:电离程度,离子密度或电子密度与总粒子数之比。
		\item 等离子体密度:单位体积内粒子的数目。$n_i$ 表示离子密度,$n_e$ 表示电子密度。
		\item 等离子体温度:对于平衡态等离子体(高温等离子体),温度是各种粒子热运动的平均量度。对于非平衡态等离子体(低温等离子体),一般用 $T_i$ 表示离子温度,$T_e$ 表示电子温度,经常用 eV 作单位。
		\item 德拜长度:等离子体中任一电荷的电场所能作用的距离。德拜长度表示等离子体能保持的最小尺度,其中包含大量带电粒子。$\lambda_D = \sqrt{\frac{\epsilon_0 n_e k T}{e^2}}$
		\item 等离子体振荡频率:等离子体中因电荷分离形成电场,由于电子和离子间的静电吸引力,形成朗谬尔振荡,频率为 $\omega = \sqrt{\frac{e n_e}{\epsilon_0 m_e}}$。
	\end{itemize}
	
	\begin{question}
	四极质谱仪中四极电场特性如何?其中离子如何运动?四极质谱仪基本原理与过程?
	\end{question}
	四极质谱仪的质量分析器由四根杆状电极组成,两对电极之间施加交变射频场,在一定频率的射频电压与直流电压作用下,只允许一定质荷比的离子通过四极分析器而到达接收器。四极质谱仪基本原理与过程包括:离子源中的电子枪发射出的电子将气体分子电离成离子或离子团,离子或离子团在电场作用下进入四极杆过滤系统,四极杆上加有在一定频率的射频电压与直流电压,使得一定质量电荷比的离子可稳定地通过四极杆质量过滤器,而不会撞上或逸出四极杆(马修方程被用于描述带电粒子在射频四极场中的运动方程及稳定区域),从而对进入四极杆区域的离子根据质量电荷比进行过滤,电控单元测量每种质荷比的离子数量,归纳成质谱。
	

\begin{question}
	为什么机械泵先抽真空压强低于 10 Pa 以下才能开启分子泵电源?
	\end{question}
	分子泵对环境条件非常敏感,工作时需要在相对较高的真空度下进行,以防止灰尘或气体对其内部部件造成损害。因此,机械泵需先将腔体的真空度降低到 10 Pa 以下,以确保分子泵的安全启动和稳定运行。
	
	\begin{question}
	为什么开启四极质谱仪时保证真空压强低于 5.0E-2 Pa?
	\end{question}
	在压强高于 5.0E-2 Pa 的情况下,离子源的灯丝容易加速老化和断裂,同时还可能导致离子源透镜结构氧化,从而降低离子化效率。因此,保持低压是确保设备稳定性和性能的关键。
	
	\begin{question}
	列举 3 个真空科学与技术应用的实例。
	\end{question}
	\begin{itemize}
		\item 低真空应用:如吸尘器和真空包装机,通过压力差进行物料的夹持和运输。
		\item 中真空应用:如真空冶金和气体排除,用于制造灯泡和热绝缘材料。
		\item 高真空应用:如电子显微镜和真空蒸镀设备,能够避免分子之间的碰撞,提高实验精度。
	\end{itemize}
	
	\begin{question}
	列举 3 个等离子体科学与技术应用的实例。
	\end{question}
	\begin{itemize}
		\item 能源领域:等离子体在核聚变中的应用,能够为人类提供可持续的清洁能源。
		\item 材料科学:等离子体处理技术可改善材料表面特性,提高耐磨性和耐腐蚀性。
		\item 医疗领域:等离子体消毒技术能有效清洁医疗器械,减少交叉感染风险,并用于治疗某些皮肤病。
	\end{itemize}
	
	% 实验记录	
	\clearpage
	\section{D1 锁相放大器与弱信号测量(1) \quad\heiti 预习报告}
	% ---
	
	\clearpage
	% 顶栏
	\begin{table}
		\renewcommand\arraystretch{1.7}
		\centering
		\begin{tabularx}{\textwidth}{|X|X|X|X|}
			\hline
			专业: & 物理学 & 年级: & 2022级 \\
			\hline
			姓名: &黄罗琳、王显  & 学号: & 22344001、22344002\\
			\hline
			室温: & 26℃ & 实验地点: & A101 \\
			\hline
			学生签名:& \includegraphics[width=1cm]{签字.jpg} \includegraphics[width=1cm]{wx.jpg} & 评分: &\\
			\hline
			实验时间:& 2024/9/20 & 教师签名:&\\
			\hline
		\end{tabularx}
	\end{table}
	% ---
	
	% 小标题
	\section{ D1 锁相放大器与弱信号测量(1)\quad\heiti 实验记录}
	% ---
	
	\clearpage

	
	% 问题记录
	\subsection{实验过程遇到问题及解决办法}
	\begin{enumerate}
		\item 
	\end{enumerate}
	% ---
	
	
	
	% 分析与讨论	
	\clearpage
	
	% 顶栏
	\begin{table}
		\renewcommand\arraystretch{1.7}
		\begin{tabularx}{\textwidth}{|X|X|X|X|}
			\hline
			专业:& 物理学 &年级:& 2022级\\
			\hline
			姓名: &  & 学号:& \\
			\hline
			日期:&  & 评分: &\\
			\hline
		\end{tabularx}
	\end{table}
	% ---
	
	% 小标题
	\section{D1 锁相放大器与弱信号测量(1)\quad\heiti 分析与讨论}
	% ---
	
	% 数据处理
	\subsection{实验数据分析}
	
	%
	\subsubsection{}
	\begin{enumerate}
		\item 
	\end{enumerate}
	
	%
	\subsubsection{}
	\begin{enumerate}
		\item 
	\end{enumerate}
	
	%
	\subsubsection{}
	
	% ---
	
	% 实验后思考题
	\subsection{实验后思考题}
	
	%思考题1
	\begin{question}
		
	\end{question}
	
	% 思考题2
	\begin{question}
		
	\end{question}
	
	% 思考题3
	\begin{question}
		
	\end{question}
	
	% ---
	
	
	% 结语部分
	\clearpage
	
	% 小标题
	\section{ETX 实验名称××× \quad\heiti 结语}
	% ---
	
	% 总结、杂谈与致谢
	\subsection{实验心得和体会、意见建议等}
	\begin{enumerate}
		\item 
	\end{enumerate}
	% ---
	

	% 附件
	\subsection{附件及实验相关的软硬件资料等}
	试验台桌面整理如%\cref{}所示。
	
	实验报告个人签名如

	% ---
	
	
\end{document}